\documentclass[oneside, 12pt]{ctexart}

\usepackage[left=2cm, right=2cm, top=2.5cm, bottom=2.5cm]{geometry} %margin设置

\usepackage{amsmath}
\usepackage{amssymb}
\usepackage{mathtools}
\usepackage{listings}
\usepackage{tikz-cd}

\author{zysx997}
\title{后继结构}

\begin{document}

\maketitle

\section{认识数,从认识数数开始}

原本打算把标题起成“认识数学,从认识数数开始的”,但仔细一想不太合理。如果想要认识自然数,从数数开始是很有必要性的。但是对于整个数学,或许认识集合、认识范畴、认识泛性质才是最为重要的。正如接下来我在处理“数数”的问题中,多次用到集合、范畴、泛性质的工具。这些工具或许不是必要的,但往往能解释清楚我们进行构造的动机,或是构造的原因。因此,认识这些概念是有益的,从“数数”这个简单的例子开始,用来辅助理解这些概念也是有帮助的。

我认为数学中定义一种结构的抽象有两个方面。一是把现实中或者概念中的直觉用数学的语言描述,给出一个“基础”,二是利用逻辑推理,在基础上附加更多“良好”的性质,以便我们对结构进行处理。在这个观点下,让我们来看看数学中是怎么对自然数进行抽象的。首先,自然数对应的现实直觉来自“数数”,运用集合代数的知识,我们可以将“自然数集”抽象为一个集合$\mathbb{N}$,然后,将“数数”这个操作定义为一个从$\mathbb{N}$到其自身的映射$s$。这样,我们就把自然数的定义从“数数”这个操作抽象为一个集合上的映射。这一步的抽象是很自然的,因为我们知道,自然数是一个集合,而“数数”这个操作就是把集合中的元素按照一定的顺序取出来。这个操作可以被抽象为一个映射。然而,我们并没有定义这个映射,也没有给出这个映射的性质。这就是我们需要进行的第二个步骤。我们需要定义这个映射,或是给出这个映射的性质。这个映射的定义或性质的给出,就是我们需要进行的第二步抽象。如果熟悉Peano公理,那么你应该会明白,Peano公理的几条性质就是第二步抽象的结果。在这里我们先列出自然数集$\mathbb{N}$与后继映射$s \colon \mathbb{N} \to \mathbb{N}$满足的Peano公理:

\begin{description}
	\item[Zero]
		\begin{gather*}
			0 \in \mathbb{N}
		\end{gather*}
	\item[Zero is not a successor]
		\begin{gather*}
			\mathop{\forall}\limits_{n \in \mathbb{N}} ns \not= 0
		\end{gather*}
	\item[Injective]
		\begin{gather*}
			\mathop{\forall}\limits_{m,n \in \mathbb{N}} ms = ns \rightarrow m = n
		\end{gather*}
	\item[Induction]
		\begin{gather*}
			\mathop{\forall}\limits_{A \subset \mathbb{N}} (0 \in A \wedge (\mathop{\forall}\limits_{n \in \mathbb{N}} n \in A \rightarrow ns \in A)) \rightarrow A = \mathbb{N}
		\end{gather*}
\end{description}

注意,在上面的符号中我们对$n$在映射$s$下的像记为$ns$,而不是$s(n)$。这样做是为了符合我们从左向右书写的习惯。另外,上面列出的Peano公理说没有包括$ns \in \mathbb{N}$这一条,因为$s$作为$\mathbb{N} \to \mathbb{N}$的函数,已经包含了这个性质。注意,Peano公理的Zero条目并非关于映射$s$,而是关于集合$\mathbb{N}$。因此,自然数集的基石与其说是集合$\mathbb{N}$与映射$s$两者,不如说是集合$\mathbb{N}$、映射$s$与元素$0$三者。而在这三者的基础上,进一步应满足Peano公理的Zero is not a successor、Injective与Induction三条性质。但是,除了抽象的第一步对$s$的定义,再加上$0$作为某个“起点”符合我们对于“数数”直觉的抽象之外,我们并不知道剩下的三条性质是如何与我们关于“数数”的直觉相联系的。或许更好的问法是:具有一个确定点$x$,一个自映射$s$的集合$X$有很多,为什么我们偏偏对于满足上面的三条性质的有确定点$0$,自映射$s$的集合$\mathbb{N}$感兴趣?这个问题,更一般的问法,是“一个结构的全体中有那么多结构对象,为什么会有一些特殊的结构对象,我们特别感兴趣?”。这个问题的解答,其主观成分在于“我们对怎样的数学对象感兴趣”,而客观成分在于,当我们确定了对怎样的数学对象兴趣后,我们能严格地证明,某些性质是这些数学对象的必然性质。这就是数学的魅力所在。

让我们来先对主观成分进行讨论。我们会对怎样的数学对象感兴趣呢?首先,这里要澄清一点,这里的数学对象是来自已经定义了的结构全体的,而这样的结构全体的定义来源于直觉,是上面说的第一步抽象。已经定义的某个结构全体,其可以有各种各样不同的结构对象,我们最为关注的是能反映所有对象性质的那些特别的对象。举例明之,如果结构全体是一个闭区间,对象是其中的元素,那么我们最关注的对象自然是最大元$b$与最小元$a$,因为如果确定了这两者,那么所有其他的元素都得到了确定。在这里,我们用特殊的两个对象,得到了关于全体对象的性质。像这样关于全体对象的性质,称为这个结构的泛性质。一个对象能反映结构的泛性质,是我们对这个对象感兴趣的主要原因。这里要说明的是,对象之所以能反映整个结构的泛性质,是因为这个对象与结构的其他对象之间有着联系,这种联系也是结构必须含有的一部分,不然的话,结构中的对象就是孤立的,没有联系,也就不可能用一个对象反映整个结构的性质。

关于问题的客观成分,需要用到范畴论。范畴论是系统地研究结构、关系、泛性质的理论。范畴论是数学的一种基础理论,它研究的是数学对象之间的关系,以及这些关系的性质。范畴论的基本概念是范畴,一个范畴包括了对象、态射与态射的复合。对象是我们研究的数学对象,态射是对象之间的联系,这种联系可以通过复合进行传递。有了范畴提供了关于结构、对象、关系的统一语言,我们就可以证明某些对象在某种意义下的唯一性,这就得到了这样的数学对象的“必然性质”。接下来,一旦确定了对象和联系的定义,我们就会用“范畴”代替“结构全体”,用“对象”代替“结构对象”,用“态射”代替“联系”,作为更标准的术语。

让我们回到关于自然数的问题。我们由第一步抽象,得到一种具有一定结构的对象:一个集合$X$,具有一个自映射$s$,同时具有一个确定点$x$。这种结构的对象包含了$3$个特征,或者说属性,我们用$3$元组方便地表示它,记为$(X, s, x)$。这里使用$X$与$x$作为符号而不使用$\mathbb{N}$与$0$的原因是,当我们提及$\mathbb{N}$与$0$时,往往默认了Peano公理,而这里我们考虑的是不一定具有特殊性的结构对象。这样的结构对象,我们就由它的元素构成,叫它“带基点的后继集合”吧。在这里,基点指的是$x$,后继指的是集合$X$上的自映射$s$,我们将其视为一种一元运算,或者也可以叫算子,称为后继算子。有了结构的对象,它们之间还需要相互联系。在集合范畴中,我们定义态射的标准方式是把态射定义为两个集合之间的映射。至于集合之间的联系为什么既不是要求更高的单射、满射、双射,又不是更弱的二元关系,这个问题有些意思。不过,我们的标准做法依然是把集合间的态射定义为映射。而我们在此处考虑的带基点的后继集合,它本质上是在一个集合上附加了两个结构:一个自映射与一个基点。因此,我们一方面继承集合范畴的态射,把带基点的后继集合$(X, s, x)$到带基点的后继集合$(X', s', x')$间的态射定义为某种映射$f \colon X \to X'$。另一方面,我们还需要考虑这个映射应该在附加的两个结构上建立联系,或者说如何保持自映射与基点的结构。保持这两个结构的方法很简单,首先是对于基点,保持基点我们能想到的唯一标准方法无非是使$xf = x'$;而对于后继算子,保持运算的标准方法是,考虑$X$中的元素$y, z$满足$ys = z$,我们可以对它在这两个点在映射$f$下得到像$yf, zf$,既然和$X$中$y$可由一步后继得到$z$,那么,我们就想在$X'$中,使像$yf$由一步后继得到$zf$。换言之,我们想让等式$yfs' = zf$对一切满足$ys = z$的$y,z \in X$成立。由于$z$可用$y$代入,这相当于下面的式子成立:

\begin{gather*}
	\mathop{\forall}\limits_{y \in X} yfs' = ysf, \quad \text{即} \quad fs' = sf.
\end{gather*}

实际上,我们可以用交换图来更加直观地展示这个关系:

\begin{center}
\begin{tikzcd}
	X \arrow[r, "f"] \arrow[d, "s"'] & X' \arrow[d, "s'"] \\
	X \arrow[r, "f"'] & X'
\end{tikzcd}
\end{center}

我们的态射定义源于集合映射定义,因此态射的复合可以直接定义为集合映射的复合,单位态射也可直接定义为一个集合的恒等映射。严格地说,我们还需验证态射的复合也是态射,关键在于“保持附加结构”的验证。这里我们使用交换图的方式进行直观验证:

\begin{center}
\begin{tikzcd}
	X \arrow[r, "f"] \arrow[d, "s"'] & X' \arrow[r, "g"] \arrow[d, "s'"'] & X'' \arrow[d, "s''"] \\
	X \arrow[r, "f"'] & X' \arrow[r, "g"'] & X''
\end{tikzcd}
\end{center}

我们给出带基点的后继集合的范畴的定义:

\begin{description}
	\item[对象] 所有带基点的后继集合$(X, s, x)$。
	\item[态射] 从$(X, s, x)$到$(X', s', x')$的态射是一个映射$f \colon X \to X'$,满足$xf = x'$与$fs' = sf$。
	\item[态射的复合] 若$f \colon (X, s, x) \to (X', s', x')$,$g \colon (X', s', x') \to (X'', s'', x'')$,则$f \circ g \colon (X, s, x) \to (X'', s'', x'')$定义为集合映射的复合$f \circ g$。
	\item[单位态射] 对于每个对象$(X, s, x)$,单位态射$\text{id}_{(X, s, x)} \colon (X, s, x) \to (X, s, x)$定义为集合$X$上的恒等映射$\text{id}_X$。
\end{description}


\end{document}