\documentclass[oneside, 12pt]{ctexart}

\usepackage[left=2cm, right=2cm, top=2.5cm, bottom=2.5cm]{geometry} %margin设置

\usepackage{amsmath}
\usepackage{amssymb}
\usepackage{mathtools}
\usepackage{listings}

\author{Zys}

\begin{document}

\title

\subsection*{认识数,从认识数数开始}

原本打算把标题起成“认识数学,从认识数数开始的”,但仔细一想不太合理。如果想要认识自然数,从数数开始是很有必要性的。但是对于整个数学,或许认识集合、认识范畴、认识泛性质才是最为重要的。正如接下来我在处理“数数”的问题中,多次用到集合、范畴、泛性质的工具。这些工具或许不是必要的,但往往能解释清楚我们进行构造的动机,或是构造的原因。因此,认识这些概念是有益的,从“数数”这个简单的例子开始,用来辅助理解这些概念也是有帮助的。

我认为数学中定义一种结构的抽象有两个方面。一是把现实中或者概念中的直觉用数学的语言描述,给出一个“基础”,二是利用逻辑推理,在基础上附加更多“良好”的性质,以便我们对结构进行处理。在这个观点下,让我们来看看数学中是怎么对自然数进行抽象的。首先,自然数对应的现实直觉来自“数数”,运用集合代数的知识,我们可以将“自然数集”抽象为一个集合$\mathbb{N}$,然后,将“数数”这个操作定义为一个从$\mathbb{N}$到其自身的映射$s$。这样,我们就把自然数的定义从“数数”这个操作抽象为一个集合上的映射。这一步的抽象是很自然的,因为我们知道,自然数是一个集合,而“数数”这个操作就是把集合中的元素按照一定的顺序取出来。这个操作可以被抽象为一个映射。然而,我们并没有定义这个映射,也没有给出这个映射的性质。这就是我们需要进行的第二个步骤。我们需要定义这个映射,或是给出这个映射的性质。这个映射的定义或性质的给出,就是我们需要进行的第二步抽象。如果熟悉Peano公理,那么你应该会明白,Peano公理的几条性质就是第二步抽象的结果。在这里我们先列出自然数集$\mathbb{N}$与后继映射$s \colon \mathbb{N} \to \mathbb{N}$满足的Peano公理:

\begin{description}
	\item[Zero]
		\begin{gather*}
			0 \in \mathbb{N}
		\end{gather*}
	\item[Zero is not a successor]
		\begin{gather*}
			\mathop{\forall}\limits_{n \in \mathbb{N}} ns \not= 0
		\end{gather*}
	\item[Injective]
		\begin{gather*}
			\mathop{\forall}\limits_{m,n \in \mathbb{N}} ms = ns \rightarrow m = n
		\end{gather*}
	\item[Induction]
		\begin{gather*}
			\mathop{\forall}\limits_{A \subset \mathbb{N}} (0 \in A \wedge (\mathop{\forall}\limits_{n \in \mathbb{N}} n \in A \rightarrow ns \in A)) \rightarrow A = \mathbb{N}
		\end{gather*}
\end{description}

注意,在上面的符号中我们对$n$在映射$s$下的像记为$ns$,而不是$s(n)$。这样做是为了符合我们从左向右书写的习惯。另外,上面列出的Peano公理说没有包括$ns \in \mathbb{N}$这一条,因为$s$作为$\mathbb{N} \to \mathbb{N}$的函数,已经包含了这个性质。注意,Peano公理的Zero条目并非关于映射$s$,而是关于集合$\mathbb{N}$。因此,自然数集的基石与其说是集合$\mathbb{N}$与映射$s$两者,不如说是集合$\mathbb{N}$、映射$s$与元素$0$三者。而在这三者的基础上,进一步应满足Peano公理的Zero is not a successor、Injective与Induction三条性质。但是,除了抽象的第一步对$s$的定义符合我们对于“数数”直觉的抽象之外,我们并不知道剩下的四条性质(包括Zero)是如何与我们关于“数数”的直觉相联系的。或许更好的问法是:我们为什么对于满足这四条性质的集合$\mathbb{N}$与映射$s$感兴趣?这个问题的答案,其主观成分在于“我们对怎样的数学对象感兴趣”,而客观成分在于,当我们确定了对怎样的数学对象兴趣后,我们能严格地证明,某些性质是这些数学对象的必然性质。这就是数学的魅力所在。

让我们来先对主观成分进行讨论。我们会对怎样的数学对象感兴趣呢?首先,这里要澄清一点,这里的数学对象是来自已经定义了的结构的,而这样的结构的定义来源于直觉,是上面说的第一步抽象。已经定义的某个结构,其可以有各种各样不同的对象,我们最为关注的是能反映所有对象性质的那些特别的对象。举例明之,如果结构是一个闭区间,对象是其中的元素,那么我们最关注的对象自然是最大元$b$与最小元$a$,因为如果确定了这两者,那么所有其他的元素都得到了确定。像这样所有对象的性质,称为这个结构的泛性质。一个对象能反映结构的泛性质,是我们对这个对象感兴趣的主要原因。对象之所以能反映整个结构的泛性质,是因为这个对象与结构的其他对象之间有某种关系,这种关系也是结构必须含有的一部分,不然的话,结构中的对象就是孤立的,没有联系,也就没有结构。系统地研究结构、关系、泛性质的理论,就是范畴论。范畴论是数学的一种基础理论,它研究的是数学对象之间的关系,以及这些关系的性质。范畴论的基本概念是范畴,一个范畴包括了对象、态射与态射的复合。对象是我们研究的数学对象,态射是对象之间的关系,态射的复合是关系的复合。

\end{document}